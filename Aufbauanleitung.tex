\documentclass[12pt, a4paper]{article}		%document style
\usepackage[utf8]{inputenc}					%input encoding
\usepackage[T1]{fontenc}					%font package

\usepackage{graphicx}						%graphics
\usepackage{wrapfig}						%graphics wrapping
\usepackage[font=small,labelfont=bf,labelsep=colon]{caption}	%caption setup
\captionsetup[figure]{name=Abb.}			%caption custom text
\usepackage{subcaption}						%subfigures
%\graphicspath{{./pics/}}					%graphics path (doesn't work yet?)

\usepackage{amsmath}						%math symbols and notation

\usepackage{multicol}						%multi-column equations

\usepackage{gensymb}						%generic symbols

\usepackage{hyperref}						%linking
\usepackage{siunitx}			% Typesetting von Wert+Einheit-Paaren (e.g. 2 cm)


%\usepackage{dingbat}	
\usepackage{float}	
%\usepackage{fontawesome}

%\interfootnotelinepenalty=10000				%penalty for splitting footnotes
%\raggedbottom								%better alignment at bottom

\usepackage[margin=1.3in]{geometry}			%page margins

\usepackage[ngerman]{babel}					%german spelling/hyphenation
%\hyphenation{Über-tra-gungs-fun-kti-on}		%explicit hyphenation

%\input{./footnotes.tex}						%footnotes
\graphicspath{./Grafiken}
\begin{document}
%-----------------------------------------------------------------------------
\title
{
  \includegraphics[scale=0.5]{Grafiken/OBS_Logo.jpg}\\
	

  %\large{
  %      OpenBikeSensor\\
  %      PCB}
      %\\*[60pt]
			%\newline
      \huge{\textbf
        {
        	Aufbauanleitung PCB\\
					}
					}
         gültig für\\*[5pt]
        00.02.05-00.02.08\\*[5pt]    
}

\author{Benjamin Bös}

\date{\today\\*[60pt]}



\maketitle         % erstellt Dokument-Kopf mit Titel und Authoren wie oben definiert
%\begin{center}
%\textbf{Allgemeines:}\\
%Ein Protokoll soll mit Hilfe eines PC und entsprechender Programme (z.B. \LaTeX, OpenOffice oder Word) angefertigt werden. Es sollten die folgenden Punkte enthalten sein!
%\end{center}
\thispagestyle{empty}

\newpage
%---------------------------------------------------------------------
%\section{Termin 2: Temperature measurement}

\section{Warnhinweise}
Achtung bitte lies die folgenden Hinweise bevor du mit dem Aufbau anfängst. Wir wissen genau wie du, dass keiner gerne Warnhinweise liest, aber gerade wenn du Anfänger bist könnte dies wichtig sein. 

\begin{itemize}
	\item Du baust den Sensor vollkommen selbstständig und ohne unsere Aufsicht auf. Du bist daher auch selbst für Fehler oder Verletzungen verantwortlich sollten welche auftreten.
	\item Gehe bewusst und gewissenhaft mit deinen Werkzeugen um. An scharfen Werkzeugen wie einem Cuttermesser oder einem Seitenschneider kannst du dich schneiden. Das vordere Ende des Lötkolbens kann bis zu \SI{450}{\degreeCelsius} heiß werden. Berühre daher immer nur das dafür vorgesehen Griffstück. Sollte er dir wegrutschen oder herunterfallen weiche daher lieber aus anstatt ihn aufzufangen.
	\item Es handelt sich hierbei um einen Bausatz und kein fertiges Gerät. Alles was wir dir mit den Bauteillisten, Schaltungsentwürfen und Anleitungen zeigen sind Vorschläge und können Fehler enthalten. Bist du dir an einer Stelle nicht sicher oder du glaubst hier könnte ein Fehler vorhanden sein, dann melde dich in unserem \href{openbikesensor.org/slack}{Slack}. Außerdem sind wir nicht für Fehler verantwortlich die du während des Aufbaus machst. Sollte etwas schief gehen und du brauchst Hilfe kannst du dich natürlich auch bei uns melden.
	\item Die Dämpfe die beim Löten durch das Verbrennen des Flussmittels (Flux) entstehen können gesundheitsschädlich sein. Atme sie daher nicht direkt ein. Du hast bei dir daheim oder in deinem Makerspace eine Lötdampfabsaugung? Dann nutze sie! Gerade bleifreies Lötzinn enthält mehr Flussmittel und ist daher während des Lötens auch deutlich gesundheitsschädlicher. Solltest du keine Absaugung haben ist daher bleihaltiges Zinn empfehlenswerter. Dieses solltest du allerdings wiederum nicht in den Mund nehmen und dir nach dem Löten die Hände waschen.
	\item Bei einem der Bauteile handelt es sich um eine LiPo\footnote{Lithium-Polymer}-Batterie. Diese Batterien sind zwar heute weit verbreitet, können allerdings bei falsche Handhabung in Brand geraten. Solltest du nicht vertraut mit LiPos oder dir noch unsicher sein, lies bitte den entsprechenden Abschnitt in der Anleitung die die Vorbereitung und den Umgang mit dem Akku erklärt! Außerdem empfehlen wir die Zelle aus einer vertrauenswürdigen Quelle zu beschaffen und nicht die billigste Zelle aus China zu bestellen.
\end{itemize}
\newpage

\section{Vorwort}

Alles was du hier siehst und liest ist zum größten Teil anhand der Version 00.02.00 erstellt worden. Leiterbahnführung und Beschriftung können sich daher an manchen Stellen leicht unterscheiden. In der Regel ist im dazugehörigen Text darauf aber hingewiesen und sollte kein Problem darstellen.\\

Außerdem gibt es zu einigen Komponenten mehrere Möglichkeiten der Bestückung oder diese zu befestigen. Lies daher erst einmal das komplette Unterkapitel, bis du auf die nächste Komponente stößt. Dann hast du einen Überblick welche Varianten es gibt und kannst anhand der Optionen die für dich passendste auswählen. Du bist von der Bauteilliste (BOM\footnote{bill of material}) hier her gekommen da du dir nicht sicher bist, wofür du bestimmte oder überhaupt welche Komponenten du brauchst? Auch dann kann es sich anbieten das entsprechende Kapitel zu überfliegen und dir so über die möglichen Optionen klar zu werden. Bilder erklären manchmal einfach besser als eine Tabelle es kann \\
Du findest Fehler oder hast Fragen? Dann melde dich entweder in unserem Slack (\href{openbikesensor.org/slack}{openbikesensor.org/slack}) oder hinterlasse einen Issue in unserem Git \href{https://github.com/Friends-of-OpenBikeSensor/OpenBikeSensor_PCB_Building-Instructions}{https://github.com/Friends-of-OpenBikeSensor/OpenBikeSensor\_PCB\_Building-Instructions}. 



\section{Vorbereitung}

\begin{figure}[H]
	\centering
		\includegraphics[width=0.95\textwidth]{Grafiken/20200726_121811.jpg}
	\caption{asd}
	\label{fig:20200726_121811}
\end{figure}

\begin{figure}[H]
	\centering
		\includegraphics[width=0.99\textwidth]{Grafiken/20200726_121402.jpg}
	\caption{günstiges Multimeter}
	\label{fig:20200726_121402}
\end{figure}

\begin{figure}[H]
	\centering
		\includegraphics[width=0.99\textwidth]{Grafiken/20200726_121444.jpg}
	\caption{Schraubstöcke}
	\label{fig:20200726_121444}
\end{figure}


\newpage
\section{Bestücken der Platine}
\subsection{Pull-Up Widerstände für I$^2$C des Displays}

\subsection{Spannungsversorgung}

\begin{minipage}[t]{0.49\textwidth}
\begin{figure}[H]
	\centering
		\includegraphics[width=0.99\textwidth]{Grafiken/20200726_121903.jpg}
	\caption{Seitenschneider vorsichtig ansetzen}
	\label{fig:20200726_121444}
\end{figure}
\end{minipage}
\begin{minipage}[t]{0.49\textwidth}
\begin{figure}[H]
	\centering
		\includegraphics[width=0.99\textwidth]{Grafiken/20200726_121920.jpg}
	\caption{zerteilte Stiftleiste}
	\label{fig:20200726_121444}
\end{figure}
\end{minipage}

\begin{figure}[H]
	\centering
		\includegraphics[width=0.99\textwidth]{Grafiken/20200726_122303.jpg}
	\caption{Insgesamt 11 Pins vorbereiten}
	\label{fig:20200726_121444}
\end{figure}

\begin{minipage}[t]{0.49\textwidth}
\begin{figure}[H]
	\centering
		\includegraphics[width=0.99\textwidth]{Grafiken/20200726_123048.jpg}
	\caption{Pins von oben einstecken}
	\label{fig:20200726_121444}
\end{figure}
\end{minipage}
\begin{minipage}[t]{0.49\textwidth}
\begin{figure}[H]
	\centering
		\includegraphics[width=0.99\textwidth]{Grafiken/20200726_123110.jpg}
	\caption{Module auflegen}
	\label{fig:20200726_121444}
\end{figure}
\end{minipage}
\newline

\subsubsection{Auflöten des Spannungsreglers ohne Pins}

Alternativ kann der Spannungsregler auch ohne Pins in SMD-Lötweise aufgelötet werden. Sollte dieser Weg gewählt werden müssen natürlich 5 Pins weniger von der Stiftleiste abgetrennt werden.\\
 Dazu als Erstes eines der Pads vorverzinnen und das Modul mit dessen Hilfe an einer Ecke ausrichten und fixieren. Diese Lötstelle muss erstmal nicht hübsch aussehen, sie soll erstmal nur mechanisch das Modul festhalten. Sitzt das Modul gerade können die verbleibenden 4 Pads angelötet und das erste Pad nachgelötet werden.

\begin{minipage}[t]{0.49\textwidth}
\begin{figure}[H]
	\centering
		\includegraphics[width=0.99\textwidth]{Grafiken/20200902_191734.jpg}
	\caption{vorverzinntes Pad r.o.}
	\label{fig:20200726_121444}
\end{figure}
\end{minipage}
\begin{minipage}[t]{0.49\textwidth}
\begin{figure}[H]
	\centering
		\includegraphics[width=0.99\textwidth]{Grafiken/20200902_191918.jpg}
	\caption{fertig angelötetes Modul}
	\label{fig:20200726_121444}
\end{figure}
\end{minipage}

Von der Gegenseite sollte die Platine nun wie in Abbildung \ref{fig:20200726_123700} aussehen. Die Pins die hier nun nach oben herausstehen können so bleiben, du kannst sie aber auch flach zum Plastik kürzen.

\begin{figure}[H]
	\centering
		\includegraphics[width=0.99\textwidth]{Grafiken/20200726_123700.jpg}
	\caption{Platine von der Oberseite}
	\label{fig:20200726_123700}
\end{figure}

\subsection{Einlöten Transistor (Q1) für das GPS-Modul}

Als nächstes wird der Transistor für das GPS-Modul eingelötet. Störe dich nicht daran, wenn der Transistor im Bild mit D1 beschriftet ist. Hier hat sich seit dem fotografierten Prototypen ein wenig etwas geändert, der Transistor wird aber in exakt der gleichen Ausrichtung an genau der selben Stelle eingelötet. \\

Zuerst wird der Transistor dafür von oben eingesteckt, die abgeflachte Seite zeigt dabei, analog zur Beschriftung, in Richtung des Lademoduls und des Spannungsreglers. Danach dreht ihr das Board um und haltet den Transistor mit einem Finger von unten fest, so dass er plan auf dem Board aufsitzt.\\

In Abbildung \ref{fig:20200726_124649} siehst du außerdem wie die Pins in Abbildung \ref{fig:20200726_123700} gekürzt aussehen.
\begin{minipage}[t]{0.49\textwidth}
\begin{figure}[H]
	\centering
		\includegraphics[width=0.99\textwidth]{Grafiken/20200726_124649.jpg}
	\caption{Q1 von oben durchstecken}
	\label{fig:20200726_124649}
\end{figure}
\end{minipage}
\begin{minipage}[t]{0.49\textwidth}
\begin{figure}[H]
	\centering
		\includegraphics[width=0.99\textwidth]{Grafiken/20200726_124658.jpg}
	\caption{Ansicht auf Q1 von oben}
	\label{fig:}
\end{figure}
\end{minipage}

Kürze nun die Beinchen in etwas 1,5-2 mm über der Board Oberfläche. Das Lademodul daneben ist mit seiner Dicke dabei eine gute Orientierung. Eine Standarddicke für Platinenmaterial ist 1,6 mm. Nun legst du das Board einfach auf deine Arbeitsunterlage und lötest zuerst ein Beinchen an. Kontrolliere nun von der Oberseite ob der Transistor gerade sitzt. Sollte er das nicht tun, nimm das Board in die Hand und übe leichten Druck auf den Transistor aus. Mit der anderen Hand nimmst du nun den Lötkolben und machst die Lötstelle erneut heiß bis der Transistor an die gewünschte Position gerutscht ist. Wenn du Lötstelle erkaltet ist wird der Transistor nun an seiner Stelle bleiben. Löte die restlichen Lötstellen an und löte bei Bedarf die erste Lötstelle, mit Hilfe von etwas frischen Lötzinn erneut nach.
\begin{minipage}[t]{0.49\textwidth}
\begin{figure}[H]
	\centering
		\includegraphics[width=0.99\textwidth]{Grafiken/20200726_124729.jpg}
	\caption{Beinchen von Q1 kürzen}
	\label{fig:}
\end{figure}
\end{minipage}
\begin{minipage}[t]{0.49\textwidth}
\begin{figure}[H]
	\centering
		\includegraphics[width=0.99\textwidth]{Grafiken/20200726_124826.jpg}
	\caption{Q1 angelötet}
	\label{fig:}
\end{figure}
\end{minipage}


\end{document}


