\documentclass[12pt, a4paper]{article}		%document style
\usepackage[utf8]{inputenc}					%input encoding
\usepackage[T1]{fontenc}					%font package

\usepackage{graphicx}						%graphics
\usepackage{wrapfig}						%graphics wrapping
\usepackage[font=small,labelfont=bf,labelsep=colon]{caption}	%caption setup
\captionsetup[figure]{name=Abb.}			%caption custom text
\usepackage{subcaption}						%subfigures
%\graphicspath{{./pics/}}					%graphics path (doesn't work yet?)

\usepackage{amsmath}						%math symbols and notation

\usepackage{multicol}						%multi-column equations

\usepackage{gensymb}						%generic symbols

\usepackage{hyperref}						%linking
\usepackage{siunitx}			% Typesetting von Wert+Einheit-Paaren (e.g. 2 cm)


%\usepackage{dingbat}	
\usepackage{float}	
%\usepackage{fontawesome}

%\interfootnotelinepenalty=10000				%penalty for splitting footnotes
%\raggedbottom								%better alignment at bottom

\usepackage[margin=1.3in]{geometry}			%page margins

\usepackage[ngerman]{babel}					%german spelling/hyphenation
%\hyphenation{Über-tra-gungs-fun-kti-on}		%explicit hyphenation

%\input{./footnotes.tex}						%footnotes
\graphicspath{./Grafiken}
\begin{document}
%-----------------------------------------------------------------------------
\title
{
  \includegraphics[scale=0.5]{Grafiken/OBS_Logo.jpg}\\
	

  %\large{
  %      OpenBikeSensor\\
  %      PCB}
      %\\*[60pt]
			%\newline
      \huge{\textbf
        {
        	Aufbauanleitung PCB\\
					}
					}
         gültig für\\*[5pt]
        00.02.05-00.02.08\\*[5pt]    
}

\author{Benjamin Bös}

\date{\today\\*[60pt]}



\maketitle         % erstellt Dokument-Kopf mit Titel und Authoren wie oben definiert
%\begin{center}
%\textbf{Allgemeines:}\\
%Ein Protokoll soll mit Hilfe eines PC und entsprechender Programme (z.B. \LaTeX, OpenOffice oder Word) angefertigt werden. Es sollten die folgenden Punkte enthalten sein!
%\end{center}
\thispagestyle{empty}

\newpage
%---------------------------------------------------------------------
%\section{Termin 2: Temperature measurement}

\section{Warnhinweise}
Achtung bitte lies die folgenden Hinweise bevor du mit dem Aufbau anfängst. Wir wissen genau wie du, dass keiner gerne Warnhinweise liest, aber gerade wenn du Anfänger bist könnte dies wichtig sein. 

\begin{itemize}
	\item Du baust den Sensor vollkommen selbstständig und ohne unsere Aufsicht auf. Du bist daher auch selbst für Fehler oder Verletzungen verantwortlich sollten welche auftreten.
	\item Gehe bewusst und gewissenhaft mit deinen Werkzeugen um. An scharfen Werkzeugen wie einem Cuttermesser oder einem Seitenschneider kannst du dich schneiden. Das vordere Ende des Lötkolbens kann bis zu \SI{450}{\degreeCelsius} heiß werden. Berühre daher immer nur das dafür vorgesehen Griffstück. Sollte er dir wegrutschen oder herunterfallen weiche daher lieber aus anstatt ihn aufzufangen.
	\item Es handelt sich hierbei um einen Bausatz und kein fertiges Gerät. Alles was wir dir mit den Bauteillisten, Schaltungsentwürfen und Anleitungen zeigen sind Vorschläge und können Fehler enthalten. Bist du dir an einer Stelle nicht sicher oder du glaubst hier könnte ein Fehler vorhanden sein, dann melde dich in unserem \href{openbikesensor.org/slack}{Slack}. Außerdem sind wir nicht für Fehler verantwortlich die du während des Aufbaus machst. Sollte etwas schief gehen und du brauchst Hilfe kannst du dich natürlich auch bei uns melden.
	\item Die Dämpfe die beim Löten durch das Verbrennen des Flussmittels (Flux) entstehen können gesundheitsschädlich sein. Atme sie daher nicht direkt ein. Du hast bei dir daheim oder in deinem Makerspace eine Lötdampfabsaugung? Dann nutze sie! Gerade bleifreies Lötzinn enthält mehr Flussmittel und ist daher während des Lötens auch deutlich gesundheitsschädlicher. Solltest du keine Absaugung haben ist daher bleihaltiges Zinn empfehlenswerter. Dieses solltest du allerdings wiederum nicht in den Mund nehmen und dir nach dem Löten die Hände waschen.
	\item Bei einem der Bauteile handelt es sich um eine LiPo\footnote{Lithium-Polymer}-Batterie. Diese Batterien sind zwar heute weit verbreitet, können allerdings bei falsche Handhabung in Brand geraten. Solltest du nicht vertraut mit LiPos oder dir noch unsicher sein, lies bitte den entsprechenden Abschnitt in der Anleitung die die Vorbereitung und den Umgang mit dem Akku erklärt! Außerdem empfehlen wir die Zelle aus einer vertrauenswürdigen Quelle zu beschaffen und nicht die billigste Zelle aus China zu bestellen.
\end{itemize}
\newpage

\section{Vorwort}

Alles was du hier siehst und liest ist zum größten Teil anhand der Version 00.02.00 erstellt worden. Leiterbahnführung und Beschriftung können sich daher an manchen Stellen leicht unterscheiden. In der Regel ist im dazugehörigen Text darauf aber hingewiesen und sollte kein Problem darstellen.\\
Die Bauteile R4 und R5 sollten sich nicht mehr auf deinem Board befinden. Anstelle von D1 sollte sich auf deinem Board mittig Q1 befinden. \\ \newline

Außerdem gibt es zu einigen Komponenten mehrere Möglichkeiten der Bestückung oder diese zu befestigen. Lies daher erst einmal das komplette Unterkapitel, bis du auf die nächste Komponente stößt. Dann hast du einen Überblick welche Varianten es gibt und kannst anhand der Optionen die für dich passendste auswählen. Du bist von der Bauteilliste (BOM\footnote{bill of material; zu deutsch Teileliste}) hier her gekommen da du dir nicht sicher bist, wofür du bestimmte oder überhaupt welche Komponenten du brauchst? Auch dann kann es sich anbieten das entsprechende Kapitel zu überfliegen und dir so über die möglichen Optionen klar zu werden. Bilder erklären manchmal einfach besser als eine Tabelle es kann \\
Du findest Fehler oder hast Fragen? Dann melde dich entweder in unserem Slack (\href{openbikesensor.org/slack}{openbikesensor.org/slack}) oder hinterlasse einen Issue in unserem Git \href{https://github.com/Friends-of-OpenBikeSensor/OpenBikeSensor_PCB_Building-Instructions}{https://github.com/Friends-of-OpenBikeSensor/OpenBikeSensor\_PCB\_Building-Instructions}. 



\section{Vorbereitung}

Bevor du anfängst mit Bestücken solltest du dir als Erstes das benötigte Werkzeug zurechtlegen. Was du währenddessen brauchen wirst (siehe Abbildung \ref{fig:20200726_121811}):

\begin{itemize}
	\item Lötkolben / Lötstation
	\item Lötzinn
	\item Seitenschneider
	\item eine Flachzange oder eine Spitzzange zum Biegen und Halten
	\item eine SMD-Pinzette (siehe Abbildung (TODO))
	\item ein Cuttermesser
	\item eine Abisolierzange
	\item eine Entlötpumpe und Entlötlitze
	\item ein Multimeter
\end{itemize}


\begin{figure}[H]
	\centering
		\includegraphics[width=0.95\textwidth]{Grafiken/20200726_121811.jpg}
	\caption{Überblick über das benötigte Werkzeug}
	\label{fig:20200726_121811}
\end{figure}

Du hast dir die in Abbildung \ref{fig:20200726_121811} abgebildeten Werkzeuge genauer angesehen und die Lötstation und das Multimeter kosten mehrere 100 Euro? Keine Sorge, du brauchst nicht so teurers Equipment. Ich habe dieses Werkzeug nur verwendet, da ich es zur Verfügung habe! Du kannst natürlich auch günstigere Werkzeuge, z.B. ein Multimeter wie in Abbildung \ref{fig:20200726_121402} verwenden. Auch die Lötstation muss keine so teure sein. Eine klare Empfehlung ist nur eine Lötstation oder -kolben mit einer einstellbaren Temperaturregelung. Du solltest mit bleihaltigem Lötzinn in einem Temperaturbereich von 300-\SI{330}{\degreeCelsius} und mit bleifreiem Lötzinn von 350-\SI{370}{\degreeCelsius} arbeiten. Lötkolben mit fester Temperaturregelung regeln meist auf \SI{450}{\degreeCelsius} und verbrennen zu schnell das im Lötzinn enthaltene Flussmittel.\\
Hilfreich kann es daher auch sein Flussmittel entweder in Form von Paste oder als Dosierstift zu haben. Hast du dies nicht zur Hand, kannst du auch mit der Entlötpumpe oder der Entlötlitze das alte Lötzinn entfernen und dann frisches Lötzinn mit frischen Flussmittel auf die Lötstelle geben. Manchmal reicht es auch schon nur ein wenig neues Lötzinn dazu zu geben um die Stelle ,,aufzufrischen''. Beachte aber, dass nicht zu viel Zinn auf deiner Lötstelle ist!


\begin{figure}[H]
	\centering
		\includegraphics[width=0.99\textwidth]{Grafiken/20200726_121402.jpg}
	\caption{günstiges Multimeter}
	\label{fig:20200726_121402}
\end{figure}

Ebenfalls hilfreich können Halterungen für die Platine sein. Das können zum einen solch günstige Spannrahmen in schwarz-blau wie in Abbildung \ref{fig:20200726_121444} sein oder auch ein solcher Kugelkopfschraubstock. Diese Schraubstöcke sind meist etwas teurer, sind aber extrem hilfreich wenn du öfter lötest. Die Platine oder auch andere Werkstücke wie Stecker lassen sich dort schnell und flexibel einspannen und in beliebigen Winkeln ausrichten. \\ \newline
Tip: richte dich nicht nach dem Werkstück aus, sondern versuche immer das Werkstück nach deiner besten Arbeitsposition auszurichten! Nur wenn du bequem sitzt und deine Werkzeuge (in diesem Fall Lötkolben, Zange/Pinzette und Lötzinn) ideal bedienen kannst erreichst du ein optimales Ergebnis! Nimm dir also die Zeit. Mit der Zeit passiert das dann ganz automatisch.

\begin{figure}[H]
	\centering
		\includegraphics[width=0.99\textwidth]{Grafiken/20200726_121444.jpg}
	\caption{Schraubstöcke}
	\label{fig:20200726_121444}
\end{figure}


\newpage
\section{Bestücken der Platine}
\subsection{Pull-Up Widerstände für I$^2$C des Displays}

Beginnen wir nun mit dem wahrscheinlich schwierigsten Bauteil für Anfänger, den SMD-Widerständen R6 und R7. Aber keine Sorge, das schaffst du und klingt schwieriger als es ist. Wir fangen gerade deshalb damit an, da das Board noch komplett leer ist und dir somit keinerlei Hindernisse im Weg sind. Du hast also allen Platz und Ruhe die du brauchst.
\\ \newline
Diese Widerstände waren zuerst nur als Backup gedacht und sind daher in SMD ausgeführt. Im Betrieb der Prototypen hat sich allerdings herausgestellt, dass sie doch benötigt werden.\\ \newline

Um den Widerstand nun anzulöten verzinne zuerst eines der Pads mit etwas Lötzinn (Abbildung \ref{fig:20200922_185326}). Es empfiehlt sich das Pad zu nehmen, an dem an wenigsten Kupfer ist und das möglichst nicht an eine große Fläche führt. Warum? Die Fläche wirkt wie ein Kühlkörper. Der Lötkolben muss dann nicht nur das Pad vorwärmen sondern auch einen Teil der großen Fläche. Dein Lötkolben hat dann wohlmöglich mehr zu kämpfen und braucht mehr Zeit bis das Pad auf Temperatur ist.\\ \newline
Nimm nun mit der SMD-Pinzette den Widerstand auf und lege ihn auf das Pad (Abbildung \ref{fig:20200922_185414}). Jetzt kannst du mit dem Lötkolben das zuvor aufgebrachte Lötzinn heiß machen und den Widerstand mit der Pinzette vorsichtig herunterdrücken (Abbildung \ref{fig:20200922_185449}). Die Lötzstelle muss dafür erst einmal nicht schön aussehen, der Widerstand soll nur in seiner gewollten Position halten.\\ \newline
Nun kannst du die zweite Seite anlöten. Wenn du damit zufrieden bist kannst du die erste Lötstelle nacharbeiten. Dies kannst du entweder tun indem du nochmal etwas Lötzinn hinzu gibst oder etwas Flussmittel aus anderer Quelle (Flussmittelpaste oder Flussmittelstift) hinzu gibst. Hast du nun zu viel Lötzinn auf der Stelle kannst du entweder versuchen mit zügigem Streichen des Lötkolbens über die Lötstelle etwas Lötzinn zu entfernen oder Entlötlitze zur Hilfe nehmen. Achte dabei darauf, dass du mit die Entlötlitze erst etwas heiß machst du eher auf die Lötstelle tumpfst anstatt zu reiben. Wenn du mit der Entlötlitze ,,rubbelst'' kannst du auch Kupfer ablösen, wenn du zu grob bist.\\ \newline
Hab keine Angst Entlötlitze oder die Entlötpumpe zu nutzen. Jeder macht mal Fehler. In Abbildung \ref{fig:20200726_184519} siehst du meine verbrauchte Menge, die ich während der gesamten Bestückung verwendet habe. Dies hat überhaupt nichts damit zu tun das man schlecht ist! Es ist ein Werkzeug, nutze es.

\begin{minipage}[t]{0.49\textwidth}
\begin{figure}[H]
	\centering
		\includegraphics[width=0.99\textwidth]{Grafiken/20200922_185326.jpg}
	\caption{ein Pad von R6 vorverzinnen}
	\label{fig:20200922_185326}
\end{figure}
\end{minipage}
\begin{minipage}[t]{0.49\textwidth}
\begin{figure}[H]
	\centering
		\includegraphics[width=0.99\textwidth]{Grafiken/20200922_185414.jpg}
	\caption{Widerstand mit Pinzette aufsetzen}
	\label{fig:20200922_185414}
\end{figure}
\end{minipage}

\begin{minipage}[t]{0.49\textwidth}
\begin{figure}[H]
	\centering
		\includegraphics[width=0.99\textwidth]{Grafiken/20200922_185449.jpg}
	\caption{Widerstand an einer Seite anlöten}
	\label{fig:20200922_185449}
\end{figure}  
\end{minipage}
\begin{minipage}[t]{0.49\textwidth}
\begin{figure}[H]
	\centering
		\includegraphics[width=0.99\textwidth]{Grafiken/20200726_184519.jpg}
	\caption{verwendete Menge Entlötlitze}
	\label{fig:20200726_184519}
\end{figure}
\end{minipage}
\newline

In den bisherigen Bildern wurde ein Widerstand in der Gehäusegröße 0805 verwendet. Dieser Widerstand ist dort auch vorgesehen, allerdings hast du vielleicht bereits gemerkt, dass die Pads relativ großzügig gestaltet sind. Es ist daher auch möglich Widerstände in der etwas größeren Gehäusegröße 1206 zu verwenden. Einen Vergleich dieser beiden Größen siehst du in den Abbildungen \ref{fig:R0805} und \ref{fig:R1206}. Wenn du dich fragst was diese Nummern für die Gehäusegrößen zu bedeuten haben, dann findest du \href{https://de.wikipedia.org/wiki/Chip-Bauform}{hier} weitere Informationen.

\begin{minipage}[t]{0.49\textwidth}
\begin{figure}[H]
	\centering
		\includegraphics[width=0.99\textwidth]{Grafiken/IMG_-vz92y3.jpg}
	\caption{Widerstand im Package 0805}
	\label{fig:R0805}
\end{figure}
\end{minipage}
\begin{minipage}[t]{0.49\textwidth}
\begin{figure}[H]
	\centering
		\includegraphics[width=0.99\textwidth]{Grafiken/IMG_l2oi9e.jpg}
	\caption{Widerstand im Package 1206}
	\label{fig:R1206}
\end{figure}
\end{minipage}

\subsection{Spannungsversorgung}

\begin{minipage}[t]{0.49\textwidth}
\begin{figure}[H]
	\centering
		\includegraphics[width=0.99\textwidth]{Grafiken/20200726_121903.jpg}
	\caption{Seitenschneider vorsichtig ansetzen}
	\label{fig:20200726_121444}
\end{figure}
\end{minipage}
\begin{minipage}[t]{0.49\textwidth}
\begin{figure}[H]
	\centering
		\includegraphics[width=0.99\textwidth]{Grafiken/20200726_121920.jpg}
	\caption{zerteilte Stiftleiste}
	\label{fig:20200726_121444}
\end{figure}
\end{minipage}

\begin{figure}[H]
	\centering
		\includegraphics[width=0.99\textwidth]{Grafiken/20200726_122303.jpg}
	\caption{Insgesamt 11 Pins vorbereiten}
	\label{fig:20200726_121444}
\end{figure}

\begin{minipage}[t]{0.49\textwidth}
\begin{figure}[H]
	\centering
		\includegraphics[width=0.99\textwidth]{Grafiken/20200726_123048.jpg}
	\caption{Pins von oben einstecken}
	\label{fig:20200726_121444}
\end{figure}
\end{minipage}
\begin{minipage}[t]{0.49\textwidth}
\begin{figure}[H]
	\centering
		\includegraphics[width=0.99\textwidth]{Grafiken/20200726_123110.jpg}
	\caption{Module auflegen}
	\label{fig:20200726_121444}
\end{figure}
\end{minipage}
\newline

\subsubsection{Auflöten des Spannungsreglers ohne Pins}

Alternativ kann der Spannungsregler auch ohne Pins in SMD-Lötweise aufgelötet werden. Sollte dieser Weg gewählt werden müssen natürlich 5 Pins weniger von der Stiftleiste abgetrennt werden.\\
 Dazu als Erstes eines der Pads vorverzinnen und das Modul mit dessen Hilfe an einer Ecke ausrichten und fixieren. Diese Lötstelle muss erstmal nicht hübsch aussehen, sie soll erstmal nur mechanisch das Modul festhalten. Sitzt das Modul gerade können die verbleibenden 4 Pads angelötet und das erste Pad nachgelötet werden.

\begin{minipage}[t]{0.49\textwidth}
\begin{figure}[H]
	\centering
		\includegraphics[width=0.99\textwidth]{Grafiken/20200902_191734.jpg}
	\caption{vorverzinntes Pad r.o.}
	\label{fig:20200726_121444}
\end{figure}
\end{minipage}
\begin{minipage}[t]{0.49\textwidth}
\begin{figure}[H]
	\centering
		\includegraphics[width=0.99\textwidth]{Grafiken/20200902_191918.jpg}
	\caption{fertig angelötetes Modul}
	\label{fig:20200726_121444}
\end{figure}
\end{minipage}

Von der Gegenseite sollte die Platine nun wie in Abbildung \ref{fig:20200726_123700} aussehen. Die Pins die hier nun nach oben herausstehen können so bleiben, du kannst sie aber auch flach zum Plastik kürzen.

\begin{figure}[H]
	\centering
		\includegraphics[width=0.99\textwidth]{Grafiken/20200726_123700.jpg}
	\caption{Platine von der Oberseite}
	\label{fig:20200726_123700}
\end{figure}

\subsection{Einlöten Transistor (Q1) für das GPS-Modul}

Als nächstes wird der Transistor für das GPS-Modul eingelötet. Störe dich nicht daran, wenn der Transistor im Bild mit D1 beschriftet ist. Hier hat sich seit dem fotografierten Prototypen ein wenig etwas geändert, der Transistor wird aber in exakt der gleichen Ausrichtung an genau der selben Stelle eingelötet. \\

Zuerst wird der Transistor dafür von oben eingesteckt, die abgeflachte Seite zeigt dabei, analog zur Beschriftung, in Richtung des Lademoduls und des Spannungsreglers. Danach dreht ihr das Board um und haltet den Transistor mit einem Finger von unten fest, so dass er plan auf dem Board aufsitzt.\\

In Abbildung \ref{fig:20200726_124649} siehst du außerdem wie die Pins in Abbildung \ref{fig:20200726_123700} gekürzt aussehen.
\begin{minipage}[t]{0.49\textwidth}
\begin{figure}[H]
	\centering
		\includegraphics[width=0.99\textwidth]{Grafiken/20200726_124649.jpg}
	\caption{Q1 von oben durchstecken}
	\label{fig:20200726_124649}
\end{figure}
\end{minipage}
\begin{minipage}[t]{0.49\textwidth}
\begin{figure}[H]
	\centering
		\includegraphics[width=0.99\textwidth]{Grafiken/20200726_124658.jpg}
	\caption{Ansicht auf Q1 von oben}
	\label{fig:}
\end{figure}
\end{minipage}

Kürze nun die Beinchen in etwas 1,5-2 mm über der Board Oberfläche. Das Lademodul daneben ist mit seiner Dicke dabei eine gute Orientierung. Eine Standarddicke für Platinenmaterial ist 1,6 mm. Nun legst du das Board einfach auf deine Arbeitsunterlage und lötest zuerst ein Beinchen an. Kontrolliere nun von der Oberseite ob der Transistor gerade sitzt. Sollte er das nicht tun, nimm das Board in die Hand und übe leichten Druck auf den Transistor aus. Mit der anderen Hand nimmst du nun den Lötkolben und machst die Lötstelle erneut heiß bis der Transistor an die gewünschte Position gerutscht ist. Wenn du Lötstelle erkaltet ist wird der Transistor nun an seiner Stelle bleiben. Löte die restlichen Lötstellen an und löte bei Bedarf die erste Lötstelle, mit Hilfe von etwas frischen Lötzinn erneut nach.
\begin{minipage}[t]{0.49\textwidth}
\begin{figure}[H]
	\centering
		\includegraphics[width=0.99\textwidth]{Grafiken/20200726_124729.jpg}
	\caption{Beinchen von Q1 kürzen}
	\label{fig:}
\end{figure}
\end{minipage}
\begin{minipage}[t]{0.49\textwidth}
\begin{figure}[H]
	\centering
		\includegraphics[width=0.99\textwidth]{Grafiken/20200726_124826.jpg}
	\caption{Q1 angelötet}
	\label{fig:}
\end{figure}
\end{minipage}


\end{document}


